% (The MIT License)
%
% Copyright (c) 2021-2022 Yegor Bugayenko
%
% Permission is hereby granted, free of charge, to any person obtaining a copy
% of this software and associated documentation files (the 'Software'), to deal
% in the Software without restriction, including without limitation the rights
% to use, copy, modify, merge, publish, distribute, sublicense, and/or sell
% copies of the Software, and to permit persons to whom the Software is
% furnished to do so, subject to the following conditions:
%
% The above copyright notice and this permission notice shall be included in all
% copies or substantial portions of the Software.
%
% THE SOFTWARE IS PROVIDED 'AS IS', WITHOUT WARRANTY OF ANY KIND, EXPRESS OR
% IMPLIED, INCLUDING BUT NOT LIMITED TO THE WARRANTIES OF MERCHANTABILITY,
% FITNESS FOR A PARTICULAR PURPOSE AND NONINFRINGEMENT. IN NO EVENT SHALL THE
% AUTHORS OR COPYRIGHT HOLDERS BE LIABLE FOR ANY CLAIM, DAMAGES OR OTHER
% LIABILITY, WHETHER IN AN ACTION OF CONTRACT, TORT OR OTHERWISE, ARISING FROM,
% OUT OF OR IN CONNECTION WITH THE SOFTWARE OR THE USE OR OTHER DEALINGS IN THE
% SOFTWARE.

\documentclass[12pt]{article}
\usepackage[T1]{fontenc}
\usepackage[tt=false,type1=true]{libertine}
\usepackage{clicks}
\usepackage{ffcode}

\title{\ff{clicks}: \LaTeX{} Package \\ for Slide Deck Animation}
\author{Yegor Bugayenko}
\date{ 2022/09/12}

\begin{document}
\pagenumbering{gobble}
\raggedbottom
\setlength{\parindent}{0pt}
\setlength{\columnsep}{32pt}
\setlength{\parskip}{6pt}

\maketitle

\section[Intro]{Introduction}

This package helps simulate animation in PDF documents. You put
some content on the page, then you ``click'' and the pages ends,
while the content gets copied to the next page. There, you add new
content. In the full-screen presentation mode this will look
like animation, similar to what you can get with MS PowerPoint.

\begin{ffcode}
\documentclass{article}
\usepackage{clicks}
\begin{document}
\print{Here is your wedding plan:}\click
\print{1. Buy a ring}\click
\print{2. Propose}\click
\print{3. Get married}\flush
\end{document}
\end{ffcode}

You can make it shorter, with the help of
\ff{\char`\\plick\{\}} and \ff{\char`\\plush\{\}} commands:

\begin{ffcode}
\documentclass{article}
\usepackage{clicks}
\begin{document}
\plick[2]{Here is your wedding plan:}
\plick[3]{1. Buy a ring}
\plick[5]{2. Propose}
\plush[1]{3. Get married}
\end{document}
\end{ffcode}

If you need to render the document without animation, just
use the \ff{static} package option.

The optional parameters at \ff{\char`\\plick\{\}} and \ff{\char`\\plush\{\}}
are the minutes: how long you are planning to stay at this animation.
The minutes are accumulated in the \ff{minutes} counter, which
you can show, for example, at your header. This will help you track
time during the presentation. The commands
\ff{\char`\\click\{\}} and \ff{\char`\\flush\{\}} also have the
same optional arguments.

More details about this package you can find
in the \ff{yegor256/clicks} GitHub repository.

If you want to add a feature or fix a bug, you are welcome
to submit an issue or make a pull request.

\end{document}
